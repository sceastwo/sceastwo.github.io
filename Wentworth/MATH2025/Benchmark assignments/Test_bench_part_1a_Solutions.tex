\documentclass{article}

\usepackage{fullpage}
\usepackage{amsmath}
\usepackage{amsfonts}
\usepackage{graphicx}
\usepackage{algorithmic}
\usepackage{xcolor}
\usepackage{framed}

\definecolor{dark_red}{rgb}{0.5,0.0,0.0}

\newcommand{\abs}[1]{\left|#1\right|}
\newcommand{\colvec}[3]{\begin{bmatrix} #1 \\ #2 \\ #3 \end{bmatrix}}
\newcommand{\mvec}[1]{\overrightarrow{\mathbf{#1}}}
\newcommand{\pvec}[1]{\overrightarrow{#1}}
\newcommand{\dr}[1]{\textcolor{dark_red}{#1}}

\title{Lines and Planes}
\date{}


\begin{document}

\maketitle

Define the points \(P(5,0,4)\); \(Q(1,1,1)\); \(R(7,3,1)\); and \(S(-5,-1,1)\). 

\section*{Question 1:}

\subsection*{part 1a:}

Compute the displacement of \(Q\) relative \(P\): \(\pvec{PQ}\).

\dr{\[\pvec{PQ} = \colvec{1 - 5}{1 - 0}{1 - 4} = \colvec{-4}{1}{-3}\]}


\subsection*{part 1b:}

Compute the magnitude \(\abs{\pvec{PQ}}\).

\dr{\[\abs{\pvec{PQ}} = \abs{\colvec{-4}{1}{-3}} = \sqrt{16 + 1 + 9} = \sqrt{26}\]}


\subsection*{part 1c:}

Compute a unit vector that shares the same direction of \(\pvec{PQ}\).

\dr{\[\frac{\pvec{PQ}}{\abs{\pvec{PQ}}} = \frac{1}{\sqrt{26}}\colvec{-4}{1}{-3} = \colvec{-4/\sqrt{26}}{1/\sqrt{26}}{-3/\sqrt{26}}\]}


\subsection*{part 1d:}

Compute parametric and implicit equations of a line \(L_{PQ}\) that contains \(P\) and \(Q\).

\dr{Parametric equation: }

\dr{Starting at \(P\) with direction vector \(\pvec{PQ}\), one possible parameterization of \(L_{PQ}\) is:}

\dr{\(\mathbf{r}(t) = \colvec{x(t)}{y(t)}{z(t)} = \colvec{5}{0}{4} + t\colvec{-4}{1}{-3}\) which is equivalent to \(\left\{\begin{array}{rl} x(t) = & 5 - 4t \\ y(t) = & t \\ z(t) = & 4 - 3t \end{array}\right.\)} 

\dr{Implicit equations: }

\dr{The parameterization gives \(\left\{\begin{array}{rl} t = & (x - 5)/(-4) \\ t = & y \\ t = & (z - 4)/(-3) \end{array}\right.\)} 

\dr{Since all \(t\)'s must be equal, the implicit equations are: \(\frac{x - 5}{-4} = y = \frac{z - 4}{-3}\)}



\section*{Question 2:}

\subsection*{part 2a:}

Compute the angle \(\angle RPQ\) using the dot product. 

\dr{\(\pvec{PQ} = \colvec{-4}{1}{-3}\) and \(\pvec{PR} = \colvec{7 - 5}{3 - 0}{1 - 4} = \colvec{2}{3}{-3}\) so 
\(\pvec{PQ} \cdot \pvec{PR} = -8 + 3 + 9 = 4\)}

\dr{\(\pvec{PQ} \cdot \pvec{PR} = \abs{\pvec{PQ}} \abs{\pvec{PR}} \cos(\angle RPQ)\) gives}

\dr{\begin{align*}
\angle RPQ = & \arccos\left(\frac{\pvec{PQ} \cdot \pvec{PR}}{\abs{\pvec{PQ}} \abs{\pvec{PR}}}\right) 
= \arccos\left(\frac{4}{\sqrt{16 + 1 + 9} \cdot \sqrt{4 + 9 + 9}}\right) 
= \arccos\left(\frac{4}{\sqrt{26} \cdot \sqrt{22}}\right) 
\end{align*}}


\subsection*{part 2b:}

Compute the area of triangle \(\Delta PQR\) using the cross product. 

\dr{\(\pvec{PQ} = \colvec{-4}{1}{-3}\) and \(\pvec{PR} = \colvec{2}{3}{-3}\) so 
\(\pvec{PQ} \times \pvec{PR} = \colvec{(1)(-3) - (-3)(3)}{(-3)(2) - (-4)(-3)}{(-4)(3) - (1)(2)} = \colvec{-3 + 9}{-6 - 12}{-12 - 2} = \colvec{6}{-18}{-14}\)}

\dr{The area is:
\begin{align*}
\frac{1}{2}\abs{\pvec{PQ} \times \pvec{PR}} = & \frac{1}{2}\sqrt{36 + 324 + 196} 
= \frac{\sqrt{556}}{2}
\end{align*}}


\subsection*{part 2c:}

Compute an implicit equation of a plane \(M_{PQR}\) that contains \(P\), \(Q\) and \(R\).

\dr{The cross product \(\mathbf{n} = \pvec{PQ} \times \pvec{PR} = \colvec{6}{-18}{-14}\) is a normal vector to \(M_{PQR}\).}  

\dr{Using \(P(5,0,4)\) as the point on the plane, the implicit equation is:
\begin{align*}
& \mathbf{n} \cdot \colvec{x}{y}{z} = \mathbf{n} \cdot \colvec{5}{0}{4} 
\iff 6x - 18y - 14z = 30 + 0 - 56 
\iff 6x - 18y - 14z = -26 \\
\iff & 3x - 9y - 7z = -13
\end{align*}}

\dr{Therefore \(M_{PQR}\) has the implicit equation: \(3x - 9y - 7z = -13\)}


\subsection*{part 2d:}

Find the shortest distance between \(R\) and the line \(L_{PQ}\) that contains \(P\) and \(Q\).

\dr{The shortest distance is:
\begin{align*}
d = & \abs{\textbf{perp}_{\pvec{PQ}}(\pvec{PR})} 
= \frac{\abs{\pvec{PQ} \times \pvec{PR}}}{\abs{\pvec{PQ}}} 
= \frac{\sqrt{36 + 324 + 196}}{\sqrt{16 + 1 + 9}} 
= \frac{\sqrt{556}}{\sqrt{26}}
\end{align*}}


\subsection*{part 2e:}

Find the intersection between the line \(L_{PQ}\) that contains \(P\) and \(Q\) and the line \(L_{RS}\) that contains \(R\) and \(S\), if this intersection exists.

\dr{From question 1d, line \(L_{PQ}\) has the parameterization \(\left\{\begin{array}{rl} x(t) = & 5 - 4t \\ y(t) = & t \\ z(t) = & 4 - 3t \end{array}\right.\)} 

\dr{Via similar steps, line \(L_{RS}\) has the parameterization \(\left\{\begin{array}{rl} x(t) = & 7 - 12t \\ y(t) = & 3 - 4t \\ z(t) = & 1 \end{array}\right.\)}

\dr{Let \(t_1\) be the parameter value for \(L_{PQ}\), and \(t_2\) be the parameter value for \(L_{RS}\) such that the same point is generated by both lines. This gives the 3 equations: 
\[\left\{\begin{array}{rl} 5 - 4t_1 = & 7 - 12t_2 \\ t_1 = & 3 - 4t_2 \\ 4 - 3t_1 = & 1 \end{array}\right.\]}

\dr{Solving the top equation gives \(5 - 4t_1 = 7 - 12t_2 \iff 12t_2 = 2 + 4t_1 \iff t_2 = 1/6 + (1/3)t_1\). Eliminating \(t_2\) in the second equation gives \(t_1 = 3 + (-2/3 - (4/3)t_1) \iff (7/3)t_1 = 7/3 \iff t_1 = 1\), which substituting into \(t_2 = 1/6 + (1/3)t_1\) gives \(t_2 = 1/6 + 1/3 = 1/2\).}

\dr{Lastly, substituting into the bottom equation yields \(1 = 1\). This means that there \textbf{is} a solution and that the lines \textbf{intersect}. If a contradiction, like \(1 = 2\), was attained, then there would be no solution and no intersection.}

\dr{Using \(t_1 = 1\) and \(t_2 = 1/2\), \(L_{PQ}\) and \(L_{RS}\) intersect at \((1,1,1)\).}


\subsection*{part 2f:}

Find the volume of the parallelepiped bounded by \(\pvec{PQ}\), \(\pvec{PR}\), and \(\pvec{PS}\). Explain your result.

\dr{The volume is:
\begin{align*}
\abs{\pvec{PS} \cdot (\pvec{PQ} \times \pvec{PR})} 
= & \abs{\colvec{-10}{-1}{-3} \cdot \colvec{6}{-18}{-14}} 
= \abs{-60 + 18 + 42} 
= 0
\end{align*}}

\dr{The 0 volume means that the parallelepiped is flat and that \(P\), \(Q\), \(R\), and \(S\) all lie in the same plane.}



\section*{Question 3:}

\subsection*{part 3a:}

Given the line \(L_1: \left\{\begin{array}{rl} x(t) = & 1 + t \\ y(t) = & 2 + 3t \\ z(t) = & 1 + t \end{array}\right.\) and the plane \(M_1: 2x - y + 7z = 9\), find the intersection between \(L_1\) and \(M_1\).

\dr{Finding the value of parameter \(t\) in line \(L_1\) that will generate a point in plane \(M_1\) requires that \(x(t) = 1 + t\); \(y(t) = 2 + 3t\); and \(z(t) = 1 + t\) satisfy \(2x(t) - y(t) + 7z(t) = 9\).}

\dr{\begin{align*}
& 2(1+t) - (2+3t) + 7(1+t) = 9  
\iff (2 - 3 + 7)t + (2 - 2 + 7) = 9 
\iff 6t = 2 
\iff t = 1/3
\end{align*}}

\dr{\(t = 1/3\) generates the intersection point \((4/3, 3, 4/3)\)}


\subsection*{part 3b:}

Given the plane \(M_2: x + y + z = 3\), find the intersection between \(M_1\) and \(M_2\).

\dr{The set of points that satisfy the equations \(\left\{\begin{array}{rl} 2x - y + 7z = & 9 \\ x + y + z = & 3 \end{array}\right.\) form the intersection.}

\dr{The top equation gives \(z = 9/7 - (2/7)x + (1/7)y\), which when substituted into the bottom equation gives \(x + y + (9/7 - (2/7)x + (1/7)y) = 3 \iff (5/7)x + (8/7)y = 12/7 \iff y = 3/2 - (5/8)x\). Substituting the expression for \(y\) into \(z = 9/7 - (2/7)x + (1/7)y\) gives \(z = 9/7 - (2/7)x + ((3/14) - (5/56)x) = 21/14 - (21/56)x = 3/2 - (3/8)x\).}

\dr{Letting \(x = t\) yields the parameterization of the intersection \(\left\{\begin{array}{rl} x(t) = & t \\ y(t) = & 3/2 - (5/8)t \\ z(t) = & 3/2 - (3/8)t \end{array}\right.\)}


\subsection*{part 3c:}

Given the point \(T(3,3,3)\), find the closest distance between \(T\) and \(M_1\). 

\dr{The normal vector to \(M_1\) is \(\mathbf{n} = \colvec{2}{-1}{7}\)}

\dr{The closest distance is: 
\begin{align*}
d = & \frac{\abs{\mathbf{n} \cdot \colvec{3}{3}{3} - 9}}{\abs{\mathbf{n}}} 
= \frac{\abs{6 - 3 + 21 - 9}}{\sqrt{4 + 1 + 49}} 
= \frac{15}{\sqrt{54}}
\end{align*}} 


\subsection*{part 3d:} 

The plane \(M_3: -6x + 3y - 21z = -33\) is parallel to \(M_1\). Find the separation between \(M_1\) and \(M_3\). 

\dr{The equation of \(M_1\) is \(2x - y + 7z = 9\), and the equation of \(M_3\) is equivalent to \(2x - y + 7z = 11\). The left hand side of both equations are now equal with the same normal vector \(\mathbf{n} = \colvec{2}{-1}{7}\).} 

\dr{The perpendicular separation is:
\begin{align*}
d = & \frac{\abs{11 - 9}}{\abs{\mathbf{n}}} 
= \frac{2}{\sqrt{4 + 1 + 49}} 
= \frac{2}{\sqrt{54}} 
\end{align*}}



\section*{Question 4:}

\subsection*{part a)}

Given vector \(\mathbf{u} = \colvec{2}{5}{-7}\) and vector \(\mathbf{v} = \colvec{x}{y}{1}\), find \(x\) and \(y\) such that \(\mathbf{u} || \mathbf{v}\).

\dr{\(\mathbf{u} || \mathbf{v}\) if and only if the ratios between corresponding components are all equal: \(x/2 = y/5 = 1/(-7)\). This gives \(x = -2/7\) and \(y = -5/7\).}


\subsection*{part b)}

Given vector \(\mathbf{u} = \colvec{2}{5}{-3}\) and vector \(\mathbf{v} = \colvec{-1}{1}{z}\), find \(z\) such that \(\mathbf{u} \perp \mathbf{v}\).

\dr{\(\mathbf{u} \perp \mathbf{v}\) if and only if \(\mathbf{u} \cdot \mathbf{v} = 0\) which yields \(-2 + 5 - 3z = 0 \iff z = 1\)}














\end{document}