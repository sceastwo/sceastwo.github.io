\documentclass{article}

\usepackage{fullpage}
\usepackage{amsmath}
\usepackage{amsfonts}
\usepackage{graphicx}
\usepackage{algorithmic}
\usepackage{xcolor}
\usepackage{framed}

\definecolor{dark_red}{rgb}{0.5,0.0,0.0}

\newcommand{\abs}[1]{\left|#1\right|}
\newcommand{\mvec}[1]{\overrightarrow{\mathbf{#1}}}
\newcommand{\pvec}[1]{\overrightarrow{#1}}
\newcommand{\colvec}[3]{\begin{bmatrix} #1 \\ #2 \\ #3 \end{bmatrix}}
\newcommand{\dr}[1]{\textcolor{dark_red}{#1}}

\title{Line Intersections}
\date{}


\begin{document}

\maketitle

Given vectors \(\mathbf{u}\) and \(\mathbf{v}\),
{%\Large
\[\textbf{proj}_{\mathbf{u}}(\mathbf{v}) = \frac{\mathbf{u} \cdot \mathbf{v}}{\abs{\mathbf{u}}^2}\mathbf{u}\]
\[\abs{\textbf{proj}_{\mathbf{u}}(\mathbf{v})} = \frac{\abs{\mathbf{u} \cdot \mathbf{v}}}{\abs{\mathbf{u}}}\]
\[\textbf{perp}_{\mathbf{u}}(\mathbf{v}) = \mathbf{v} - \textbf{proj}_{\mathbf{u}}(\mathbf{v})\]
\[\abs{\textbf{perp}_{\mathbf{u}}(\mathbf{v})} = \frac{\abs{\mathbf{u} \times \mathbf{v}}}{\abs{\mathbf{u}}}\]
}
\(\mathbf{u}||\mathbf{v}\) denotes that \(\mathbf{u}\) and \(\mathbf{v}\) are \textbf{parallel}. \\
\(\mathbf{u}\perp\mathbf{v}\) denotes that \(\mathbf{u}\) and \(\mathbf{v}\) are \textbf{perpendicular/orthogonal}.


\section*{Classifying the interaction between two lines}

Given two straight lines \(L_1\) and \(L_2\), there are 4 possible relationships between these lines:
\begin{itemize}
\item \(L_1\) and \(L_2\) are \textbf{equivalent}. 
\item \(L_1\) and \(L_2\) are \textbf{parallel but not equal}. 
\item \(L_1\) and \(L_2\) \textbf{intersect} at a single point. 
\item \(L_1\) and \(L_2\) are \textbf{skew}.
\end{itemize}

The relationship between \(L_1\) and \(L_2\) can be determined via the following algorithm:
\(L_1\) will be described by the parametric line: 
\[\mathbf{r}(t) = \mathbf{r}_{0,1} + t\mathbf{v}_1\] 
and \(L_2\) will be described by the parametric line: 
\[\mathbf{r}(t) = \mathbf{r}_{0,2} + t\mathbf{v}_2\]

\pagebreak

\begin{framed}
\begin{algorithmic}
\IF{\(\mathbf{v}_1 || \mathbf{v}_2\)}
	\STATE Let \(d = \abs{\textbf{perp}_{\mathbf{v}_1}(\mathbf{r}_{0,2}-\mathbf{r}_{0,1})} = \frac{\abs{\mathbf{v}_1 \times (\mathbf{r}_{0,2}-\mathbf{r}_{0,1})}}{\abs{\mathbf{v}_1}}\)
	\IF{\(d = 0\)}
		\STATE \(L_1\) and \(L_2\) are \textbf{equivalent}
	\ELSE
		\STATE \(L_1\) and \(L_2\) are \textbf{parallel but not equal}, and have a separation of \(d\). 
	\ENDIF
\ELSE
	\STATE Let \(\mathbf{n} = \mathbf{v}_1 \times \mathbf{v}_2\)
	\STATE Let \(d = \abs{\textbf{proj}_{\mathbf{n}}(\mathbf{r}_{0,2}-\mathbf{r}_{0,1})} = \frac{\abs{\mathbf{n} \cdot (\mathbf{r}_{0,2}-\mathbf{r}_{0,1})}}{\abs{\mathbf{n}}}\)
	\IF{\(d = 0\)}
		\STATE \(L_1\) and \(L_2\) \textbf{intersect} at a single point
	\ELSE
		\STATE \(L_1\) and \(L_2\) are \textbf{skew}, and have a closest distance of \(d\). 
	\ENDIF
\ENDIF
\end{algorithmic}
\end{framed}

For the following pairs of lines, determine if these lines are \textbf{equivalent}; are \textbf{parallel but not equal}; \textbf{intersect} at a single point; or are \textbf{skew}. For parallel lines, give the separation, and for skew lines, give the closest distance. \textbf{Show all of your work.}


%%%%%%%%%%% PAIR 1
\begin{description}
\item[Line pair 1:] ~ \\
\(L_1\) is parameterized by \(\left\{\begin{array}{rl} x(t) = & 4 - t \\ y(t) = & -4 + 2t \\ z(t) = & -2 + 5t \end{array}\right.\) and
\(L_2\) is parameterized by \(\left\{\begin{array}{rl} x(t) = & -1 - 2t \\ y(t) = & 6 + 4t \\ z(t) = & 23 + 10t \end{array}\right.\)
%\(L_1\) contains all points that satisfy the equations \(\frac{x-4}{-1} = \frac{y+4}{2} = \frac{z+2}{5}\) \\
%\(L_2\) contains all points that satisfy the equations \(\frac{x+1}{-2} = \frac{y-6}{4} = \frac{z-23}{10}\) 
\end{description}

\dr{\(\mathbf{r}_{0,1} = \colvec{4}{-4}{-2}\); \(\mathbf{v}_1 = \colvec{-1}{2}{5}\); and \(\mathbf{r}_{0,2} = \colvec{-1}{6}{23}\); \(\mathbf{v}_2 = \colvec{-2}{4}{10}\)}

\dr{\(\mathbf{v}_2 = 2\mathbf{v}_1\) so \(\mathbf{v}_1 || \mathbf{v}_2\), and hence \(L_1\) and \(L_2\) are parallel.}

\dr{\begin{align*}
d = & \abs{\textbf{perp}_{\mathbf{v}_1}(\mathbf{r}_{0,2} - \mathbf{r}_{0,1})} 
= \frac{\abs{\mathbf{v}_1 \times (\mathbf{r}_{0,2} - \mathbf{r}_{0,1})}}{\abs{\mathbf{v}_1}} 
= \frac{1}{\sqrt{1 + 4 + 25}}\abs{\colvec{-1}{2}{5} \times \colvec{-5}{10}{25}} \\
= & \frac{1}{\sqrt{30}}\abs{\colvec{(2)(25) - (5)(10)}{(5)(-5) - (-1)(25)}{(-1)(10) - (2)(-5)}} 
= \frac{1}{\sqrt{30}}\abs{\colvec{50 - 50}{-25 + 25}{-10 + 10}}  
= \frac{1}{\sqrt{30}}\abs{\colvec{0}{0}{0}} 
= 0 
\end{align*}}

\dr{The perpendicular distance between \(L_1\) and \(L_2\) is \(d = 0\), so \(L_1\) and \(L_2\) are \textbf{equivalent}.}


%%%%%%%%%%% PAIR 2
\begin{description}
\item[Line pair 2:] ~ \\
\(L_1\) is parameterized by \(\left\{\begin{array}{rl} x(t) = & -5 + t \\ y(t) = & -1 + 2t \\ z(t) = & 2 + t \end{array}\right.\) and
\(L_2\) is parameterized by \(\left\{\begin{array}{rl} x(t) = & -8 + 2t \\ y(t) = & -1 + t \\ z(t) = & -1 + 2t \end{array}\right.\)
%\(L_1\) contains all points that satisfy the equations \(x+5 = \frac{y+1}{2} = z-2\) \\
%\(L_2\) contains all points that satisfy the equations \(\frac{x+8}{2} = y+1 = \frac{z+1}{2}\) 
\end{description}

\dr{\(\mathbf{r}_{0,1} = \colvec{-5}{-1}{2}\); \(\mathbf{v}_1 = \colvec{1}{2}{1}\); and \(\mathbf{r}_{0,2} = \colvec{-8}{-1}{-1}\); \(\mathbf{v}_2 = \colvec{2}{1}{2}\)}

\dr{\(\mathbf{v}_1 \not{\!||}\; \mathbf{v}_2\), and hence \(L_1\) and \(L_2\) are not parallel.}

\dr{\begin{align*}
\mathbf{n} = & \mathbf{v}_1 \times \mathbf{v}_2 
= \colvec{1}{2}{1} \times \colvec{2}{1}{2} 
= \colvec{(2)(2) - (1)(1)}{(1)(2) - (1)(2)}{(1)(1) - (2)(2)} 
= \colvec{4 - 1}{2 - 2}{1 - 4} 
= \colvec{3}{0}{-3}
\end{align*}}

\dr{\begin{align*}
d = & \abs{\textbf{proj}_{\mathbf{n}}(\mathbf{r}_{0,2} - \mathbf{r}_{0,1})} 
= \frac{\abs{\mathbf{n} \cdot (\mathbf{r}_{0,2} - \mathbf{r}_{0,1})}}{\abs{\mathbf{n}}} 
= \frac{1}{\sqrt{9 + 0 + 9}}\abs{\colvec{3}{0}{-3} \cdot \colvec{-3}{0}{-3}} \\
= & \frac{\abs{-9 + 0 + 9}}{\sqrt{18}}
= 0
\end{align*}
}

\dr{The minimum distance between \(L_1\) and \(L_2\) is \(d = 0\), so \(L_1\) and \(L_2\) \textbf{intersect} at a single point.}


%%%%%%%%%%% PAIR 3
\begin{description}
\item[Line pair 3:] ~ \\
\(L_1\) is parameterized by \(\left\{\begin{array}{rl} x(t) = & -3 + 7t \\ y(t) = & 1 + 2t \\ z(t) = & 2 + 5t \end{array}\right.\) and
\(L_2\) is parameterized by \(\left\{\begin{array}{rl} x(t) = & 4 - 21t \\ y(t) = & 2 - 6t \\ z(t) = & 7 - 15t \end{array}\right.\)
%\(L_1\) contains all points that satisfy the equations \(\frac{x+3}{7} = \frac{y-1}{2} = \frac{z-2}{5}\) \\
%\(L_2\) contains all points that satisfy the equations \(\frac{x-4}{-21} = \frac{y-2}{-6} = \frac{z-7}{-15}\)
\end{description}

\dr{\(\mathbf{r}_{0,1} = \colvec{-3}{1}{2}\); \(\mathbf{v}_1 = \colvec{7}{2}{5}\); and \(\mathbf{r}_{0,2} = \colvec{4}{2}{7}\); \(\mathbf{v}_2 = \colvec{-21}{-6}{-15}\)}

\dr{\(\mathbf{v}_2 = -3\mathbf{v}_1\) so \(\mathbf{v}_1 || \mathbf{v}_2\), and hence \(L_1\) and \(L_2\) are parallel.}

\dr{\begin{align*}
d = & \abs{\textbf{perp}_{\mathbf{v}_1}(\mathbf{r}_{0,2} - \mathbf{r}_{0,1})} 
= \frac{\abs{\mathbf{v}_1 \times (\mathbf{r}_{0,2} - \mathbf{r}_{0,1})}}{\abs{\mathbf{v}_1}} 
= \frac{1}{\sqrt{49 + 4 + 25}}\abs{\colvec{7}{2}{5} \times \colvec{7}{1}{5}} \\
= & \frac{1}{\sqrt{78}}\abs{\colvec{(2)(5) - (5)(1)}{(5)(7) - (7)(5)}{(7)(1) - (2)(7)}} 
= \frac{1}{\sqrt{78}}\abs{\colvec{10 - 5}{35 - 35}{7 - 14}}  
= \frac{1}{\sqrt{78}}\abs{\colvec{5}{0}{-7}} 
= \frac{\sqrt{25 + 0 + 49}}{\sqrt{78}} 
= \frac{\sqrt{74}}{\sqrt{78}}
\end{align*}}

\dr{The perpendicular distance between \(L_1\) and \(L_2\) is \(d = \frac{\sqrt{74}}{\sqrt{78}} > 0\), so \(L_1\) and \(L_2\) are \textbf{parallel but not equal}.}


%%%%%%%%%%% PAIR 4
\begin{description}
\item[Line pair 4:] ~ \\
\(L_1\) is parameterized by \(\left\{\begin{array}{rl} x(t) = & -1 + t \\ y(t) = & 2t \\ z(t) = & 1 + 3t \end{array}\right.\) and
\(L_2\) is parameterized by \(\left\{\begin{array}{rl} x(t) = & -1 \\ y(t) = & 1 + 2t \\ z(t) = & 1 + 3t \end{array}\right.\)
\end{description}

\dr{\(\mathbf{r}_{0,1} = \colvec{-1}{0}{1}\); \(\mathbf{v}_1 = \colvec{1}{2}{3}\); and \(\mathbf{r}_{0,2} = \colvec{-1}{1}{1}\); \(\mathbf{v}_2 = \colvec{0}{2}{3}\)}

\dr{\(\mathbf{v}_1 \not{\!||}\; \mathbf{v}_2\), and hence \(L_1\) and \(L_2\) are not parallel.}

\dr{\begin{align*}
\mathbf{n} = & \mathbf{v}_1 \times \mathbf{v}_2 
= \colvec{1}{2}{3} \times \colvec{0}{2}{3} 
= \colvec{(2)(3) - (3)(2)}{(3)(0) - (1)(3)}{(1)(2) - (2)(0)} 
= \colvec{6 - 6}{0 - 3}{2 - 0} 
= \colvec{0}{-3}{2}
\end{align*}}

\dr{\begin{align*}
d = & \abs{\textbf{proj}_{\mathbf{n}}(\mathbf{r}_{0,2} - \mathbf{r}_{0,1})} 
= \frac{\abs{\mathbf{n} \cdot (\mathbf{r}_{0,2} - \mathbf{r}_{0,1})}}{\abs{\mathbf{n}}} 
= \frac{1}{\sqrt{0 + 9 + 4}}\abs{\colvec{0}{-3}{2} \cdot \colvec{0}{1}{0}} \\
= & \frac{\abs{0 - 3 + 0}}{\sqrt{13}} 
= \frac{3}{\sqrt{13}}
\end{align*}}

\dr{The minimum distance between \(L_1\) and \(L_2\) is \(d = \frac{3}{\sqrt{13}} > 0\), so \(L_1\) and \(L_2\) are \textbf{skew}.}


\end{document}



